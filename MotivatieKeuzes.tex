\documentclass[12pt]{article}

\setlength{\parindent}{0pt}
\usepackage[left=2.5cm,top=2cm,right=2.5cm,bottom=2cm,a4paper]{geometry}
\usepackage{fancyhdr}
\pagestyle{fancy}
\lhead{Team 4: TeamR2D2}
\rhead{Motivatie keuzes}
\renewcommand{\headrulewidth}{0.4pt}

\begin{document}
	We kozen voor de NI Myrio@-microcontroller omdat deze het simpelste is en voor de aard van de opdracht voor ons het efficiëntste leek. Bovendien hebben we via de NI Myrio altijd een handige visualisatie van het programma en kunnen we gebruik maken van de analoge afstandssensor die met een bereik van 10 to 80 centimeter de meest geschikte afstandssensor is.
	Door deze keuze kunnen we voor de andere sensoren enkel de analoge versie gebruiken. Aangezien de NI Myrio een @power input nodig heeft van 6-16 Volt, kan de powerbank niet gebruikt worden. Door 2 Lithium-ion @batterijen van @$3,6$ Volt te kiezen en deze in serie te schakelen is het wel mogelijk om een geschikte power-input te verkrijgen.
	Als persoonlijke touch kozen we om geen makerbeams te gebruiken. Hierdoor leek het ons het beste om een groot chassis te hebben. We kozen het rechthoekig zwart boven het universeel chassis omdat ze redelijk gelijkaardig zijn maar het universeel chassis 50 eenheden meer kost. Om dit chassis stabiel te maken kozen we ervoor om de ball caster te gebruiken samen met 2 @aangedreven wielen. Als wiel kozen we voor de wielen met een breedte van 8 millimeter en diameter van 60 millimeter, deze zijn de grootste en kunnen samen met de spacers van de ball caster ervoor zorgen dat het chassis perfect horizontaal ligt. Als aandrijving hadden we de voorkeur voor de "Micro Metal Gear Motor 50:1 HP". Deze heeft een 3 millimeter as, wat compatibel is met onze gekozen wielen en heeft een gemiddeld vermogen om de wagen te doen stoppen en opnieuw te versnellen. Omdat deze motoren niet meer beschikbaar waren wanneer wij onze bestelling konden plaatsen kozen we voor de "Micro Metal Gear Motor 100:1 HP", deze is redelijk gelijkaardig maar heeft een ander @vermogen. Samen met deze motoren kozen we 2 motorbeugels om de motoren aan het chassis te kunnen bevestigen en de "Dual Drive DRV8833" zodat we beide motoren tegelijk kunnen aansturen.
	Om eerst de wagen te testen voor we gaan solderen op de printplaat kozen we nog een breadbord, aangezien we maar 3 sensoren en 2 motoren hadden kozen we voor het kleinere formaat omdat dit ons minder budget kost.

	
	
	
	
	
\end{document}